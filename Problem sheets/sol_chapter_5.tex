\documentclass[oneside,11pt]{article}


%\usepackage[british]{babel}
\usepackage{amsmath}
\usepackage{amssymb}
\usepackage{amsthm}
\usepackage{amsfonts}
\usepackage{graphicx}
% \usepackage{bm}
%\usepackage{dsfont}
%\usepackage{titling}
%\usepackage{enumitem}
%\usepackage{titletoc}
\usepackage{lmodern}
\usepackage{textcomp} % Required by listings
%\usepackage{listings}
%\usepackage{thmtools}
%\usepackage{memhfixc}
\usepackage{hyperref}
%\usepackage[all]{xy}


\usepackage[T1]{fontenc}
%\usepackage[scaled=0.8]{DejaVuSansMono}
% \usepackage[charter]{mathdesign}
% \def\rmdefault{bch} % not scaled
% \def\ttdefault{blg}
% \usepackage[scaled=0.9]{inconsolata}
% \usepackage[scaled=0.8]{luximono}

%\nonzeroparskip
\raggedbottom

%% Shaded environment
\usepackage{framed}
\usepackage{xcolor}
\definecolor{shadecolor}{gray}{0.9}
\setlength{\FrameSep}{3pt}
\setlength{\OuterFrameSep}{3pt}


%%% Floats
%\setfloatlocations{figure}{tbhp}
%\setfloatlocations{table}{tbhp}
%\captionnamefont{\bfseries}
% \captiontitlefont{\itshape}
% \captionstyle[\centering]{\raggedright}
% \captionwidth{\linewidth} \changecaptionwidth
\renewcommand{\topfraction}{0.9}        % max fraction of floats at top
\renewcommand{\bottomfraction}{0.8}     % max fraction of floats at bottom
\setcounter{topnumber}{2}
\setcounter{bottomnumber}{2}
\setcounter{totalnumber}{4}     % 2 may work better
\setcounter{dbltopnumber}{2}    % for 2-column pages
\renewcommand{\dbltopfraction}{0.9}     % fit big float above 2-col. text
\renewcommand{\textfraction}{0.07}      % allow minimal text w. figs
\renewcommand{\floatpagefraction}{0.7}  % require fuller float pages
\renewcommand{\dblfloatpagefraction}{0.7}       % require fuller float pages


%%% Commands
\DeclareMathOperator*{\E}{{\mathsf E}} % expectation
\DeclareMathOperator*{\Var}{{\mathsf Var}} % variance
\DeclareMathOperator*{\Cov}{{\mathsf Cov}} % covariance
\DeclareMathOperator*{\argmin}{argmin} % argmin
\DeclareMathOperator*{\argmax}{argmax} % argmax
\newcommand{\iid}{{{}\mathbin{\stackrel{\mathsf{iid}}{\sim}}{}}}
\renewcommand{\Pr}{{\mathbb{P}}}
\newcommand{\half}{{\frac 12}}
\newcommand{\ud}{{\,\mathrm{d}}}
\newcommand{\xbf}{{\mathbf{x}}}
\newcommand{\Xbf}{{\mathbf{X}}}

%%% Theorems
%\renewenvironment{proof}[1][Proof]{\noindent{\bfseries #1.}\hspace*{1em}}{\qed\\}
%\declaretheoremstyle[spaceabove=10pt,spacebelow=3pt,headfont=\normalfont\bfseries,
%bodyfont=\normalfont,postheadspace=1em,headpunct={.\newline}]{clearprint}
%\declaretheoremstyle[spaceabove=10pt,spacebelow=3pt,headfont=\normalfont\bfseries,
%bodyfont=\normalfont,postheadspace=1em,headpunct={.}]{clearexc}
%\declaretheoremstyle[spaceabove=10pt,spacebelow=3pt,headfont=\normalfont\bfseries,
%bodyfont=\normalfont,postheadspace=1em,headpunct={.},qed=$\blacktriangleright$]{clearexm}
%\declaretheoremstyle[spaceabove=10pt,headpunct={.\newline},
%headfont={\normalfont\itshape}]{remarkmy}
%%
%\declaretheorem[style=clearprint,within=chapter,name=Definition]{definition}
%\declaretheorem[style=clearexm,within=chapter,name=Example]{example}
%\declaretheorem[style=clearexc,within=chapter,name=Exercise]{exercise}
%\declaretheorem[style=clearprint,within=chapter,name=Theorem]{theorem}
%\declaretheorem[style=clearprint,name=Lemma,sibling=theorem]{lemma}
%\declaretheorem[style=clearprint,name=Proposition,sibling=theorem]{proposition}
%\declaretheorem[style=clearexc,numbered=no,name=Corollary]{corollary}
%\declaretheorem[style=clearexc,numbered=no,name=Remark]{remark}

\renewcommand{\labelitemi}{$\bullet$}
\setlength{\emergencystretch}{.5ex}

% \hypersetup{
%   colorlinks,%
%   citecolor=black,%
%   filecolor=black,%
%   linkcolor=black,%
%   urlcolor=black,
%   linktoc=none
% }

% Page style
\pagestyle{plain}

%%% Page size
% \setstocksize{297mm}{210mm} % A4 paper
% \settypeblocksize{258mm}{160mm}{*} % A4 paper
%\setstocksize{10.3in}{7.3in}
% \settypeblocksize{217mm}{140mm}{*}
%\settypeblocksize{230mm}{160mm}{*}
%\settrimmedsize{\stockheight}{\stockwidth}{*}
%\setlength{\trimtop}{0pt}
%\setlength{\trimedge}{\stockwidth}
%\addtolength{\trimedge}{-\paperwidth}
%\setulmargins{*}{*}{1.5}
%\setlrmargins{*}{*}{1}
\addtolength{\footskip}{.25cm}
%\checkandfixthelayout





\setcounter{tocdepth}{2}

\allowdisplaybreaks

\usepackage{xr}
\externaldocument[lec1:]{../1_sampling/lec1}
\externaldocument[lec2:]{../2_estimation/lec2}
\externaldocument[lec3:]{../3_hypothesis/lec3}



\begin{document}

Chapter 5 Solutions

\textbf{Confidence intervals}

\begin{enumerate}
\item
  \begin{enumerate}
  \item Let $X_1,\ldots,X_n \iid \mathrm{N}(\mu,\sigma^2)$ where both $\mu$
    and $\sigma^2$ are unknown parameters. Using the fact that, the sample
    variance, $S^2$, is distributed as
    $(n-1)S^2/\sigma^2 \sim \mathcal{X}^2_{n-1}$, derive a level $1-\alpha$
    confidence interval for $\sigma^2$. ($\mathcal{X}^2_{k}$ denotes the
    \emph{chi-squared distribution with $k$ degrees of freedom}, which is
    available in python as \verb|scipy.stats.chi2|. It is a special case of
    the gamma distribution with shape $k/2$ and rate $1/2$.)
    
    \begin{shaded}
      We know that $\dfrac{(n-1)S^2}{\sigma^2} \sim \mathcal{X}^2_{n-1}$ so
      $Y = \dfrac{(n-1)S^2}{\sigma^2}$ is a pivot function because its
      distribution does not depend on the unknown parameter $\sigma^2$. Let
      $\chi^2_{\alpha/2}$ and $\chi^2_{1-\alpha/2}$ denote the $\alpha/2$ and
      $1-\alpha/2$ left quantiles of the $\mathcal{X}^2_{n-1}$ distribution
      respectively. Then we have the following inequality which holds with
      probability $1-\alpha$
      \begin{align*}
        & \chi^2_{\alpha/2} < Y < \chi^2_{1-\alpha/2} \\
        & \Rightarrow \chi^2_{\alpha/2} < \dfrac{(n-1)S^2}{\sigma^2} <
        \chi^2_{1-\alpha/2} \\
        & \Rightarrow \dfrac{1}{\chi^2_{1-\alpha/2}} <
        \dfrac{\sigma^2}{(n-1)S^2} < \dfrac{1}{\chi^2_{\alpha/2}} \\
        & \Rightarrow \dfrac{(n-1)S^2}{\chi^2_{1-\alpha/2}} <
        \sigma^2 < \dfrac{(n-1)S^2}{\chi^2_{\alpha/2}},
      \end{align*}
      so a level $1-\alpha$ confidence interval is
      $\left[ \dfrac{(n-1)S^2}{\chi^2_{1-\alpha/2}},
        \dfrac{(n-1)S^2}{\chi^2_{\alpha/2}} \right]$
    \end{shaded}
    
  \item Suppose that the following data were observed
    \begin{center}
      $-$1.90, $-$0.89, $-$0.87, $-$0.65, $-$0.32, $-$0.25,  0.90,  1.00,  1.18.
    \end{center}
    For these data $n=9$, $\bar x = -0.2$, and $S^2 = 1.073$. Calculate a
    95\% confidence interval for $\sigma^2$.

    \begin{shaded}
      From Python, we have
\begin{verbatim}
>>> import scipy.stats
>>> scipy.stats.chi2.ppf([.025,.975],8)
array([ 2.17973075, 17.53454614])
\end{verbatim}
      so $\chi^2_{\alpha/2} = 2.18$ and $\chi^2_{1-\alpha/2} = 17.53$. Then,
      \begin{align*}
        L&= \frac{(9-1)(1.073)}{17.53} = 0.49 \\
        U&= \frac{(9-1)(1.073)}{2.18} = 3.94,
      \end{align*}
      so the 95\% confidence interval is $[0.49,3.94]$.
    \end{shaded}
\end{enumerate}


\item
  \begin{enumerate}
  \item Let $X_1,\ldots,X_n \iid \mathrm{U}(0,\theta)$, $\theta > 0$. By
    considering an appropriate pivot construct a level $1-\alpha$
    confidence interval for $\theta$. \textit{Hint.} Let
    $W_i = X_i/\theta$.

    \begin{shaded}
      Let $W_i = X_i/\theta$. Then $W_i \sim U(0,1)$ so if we let
      $W_{(n)} = \max \{ W_1,\ldots, W_n \}$, then the distribution of
      $W_{(n)}$ does not depend on $\theta$. Note that by letting
      $X_{(n)} = \max \{ X_1,\ldots, X_n \}$, $W_{(n)} = X_{(n)}/\theta$.

      The CDF of $W_{(n)}$ is
      \begin{align*}
        F_W(w)
        & = \Pr (W_{(n)} \leq w) \\
        &= \Pr( \max \{ W_1,\ldots, W_n \} \leq w) \\
        &= \Pr( W_1 \leq w, \ldots, W_n \leq w) \\
        &= \Pr( W_1 \leq w)  \ldots \Pr( W_n \leq w),\ \text{by independence} \\
        &= w \times  \ldots \times w \\
        &= w^n
      \end{align*}
      So if we let $c_1 = (\alpha/2)^{\frac 1n}$ and
      $c_2 = (1-\alpha/2)^{\frac 1n}$, with probability $1-\alpha$,
      \begin{align*}
        & c_1 < W_{(n)} < c_2 \\
        & c_1 < X_{(n)}/\theta < c_2 \\
        & \dfrac{1}{c_2} < \theta/X_{(n)} < \dfrac{1}{c_1} \\
        & \dfrac{X_{(n)}}{c_2} < \theta < \dfrac{X_{(n)}}{c_1},
      \end{align*}
      so a level $1-\alpha$ confidence interval is
      $\left[ \dfrac{X_{(n)}}{c_2}, \dfrac{X_{(n)}}{c_1} \right]$
    \end{shaded}
    
  \item Suppose that the following data were observed
    \begin{center}
      0.90,  1.00,  1.18, 1.90, 2.20.
    \end{center}
    Calculate a 95\% confidence interval for $\theta$. Does the confidence
    interval contain the maximum likelihood estimator for $\theta$?

    \begin{shaded}
      For the given data, $n=5$ and $\max{x} = 2.20$.

      For 95\% confidence, $\alpha = 0.05$, so $c_1 = (0.025)^(1/5) =
      0.48$, and $c_2 = (0.975)^(1/5) = 0.99$. Then,
      \begin{align*}
        L&=\frac{2.20}{0.99} = 2.22 \\
        U&=\frac{2.20}{0.48} = 4.58,
      \end{align*}
      so the confidence interval is $[2.22,4.58]$.

      We can see that the confidence interval does not contain the maximum
      likelihood estimator, which equals 2.20. The confidence interval
      provides a more sensible range of values than the maximum likelihood
      estimator, which underestimates $\theta$ for this model as we have
      observed in the previous chapter.
    \end{shaded}
  \end{enumerate}
\end{enumerate}

\textbf{Hypothesis testing}

\begin{enumerate}
\item The author of a weight-loss diet claims that an average adult,
  weighting 100\,Kg, who follows the proposed diet, will lose 20\,Kg after
  1 month. What are the null and alternative hypotheses?

  \begin{shaded}
    Let $\mu$ denote the average person's weight after one month of diet.
    Then \\ H$_0$: $\mu = 100$ vs H$_1$: $\mu = 80$.
  \end{shaded}
  
\item The author of a weight-loss diet claims that an average adult,
  weighting 100\,Kg, who follows the proposed diet, will lose weight after
  1 month. What are the null and alternative hypotheses?
  \begin{shaded}
    Let $\mu$ denote the average person's weight after one month of diet.
    Then \\ H$_0$: $\mu = 100$ vs H$_1$: $\mu < 100$.
  \end{shaded}

\item The author of a weight-loss diet claims that an average adult,
  weighting 100\,Kg, who follows the proposed diet, will notice a change in
  their weight after 1 month. What are the null and alternative hypotheses?
  \begin{shaded}
    Let $\mu$ denote the average person's weight after one month of diet.
    Then \\ H$_0$: $\mu = 100$ vs H$_1$: $\mu \neq 100$.
  \end{shaded}

\item The author of a weight-loss diet claims that an average adult,
  weighting 100\,Kg, who follows the proposed diet, will lose weight after
  1 month. An experiment was conducted to verify this claim. Three adults,
  who weighted 100\,Kg, followed the diet for one month and their weights
  at the end of the month were recorded. The experimenters
  would accept the author's claim if the sample mean $\bar X$ of the three
  measured weights is less than 90. Suppose that the population standard
  deviation is $\sigma=15$. 

  \begin{enumerate}
  \item The three people's weights after the end of the month were: 82, 86,
    and 93. What is the experimenters' conclusion?
    \begin{shaded}
      In this case $\bar x = (82 + 86 + 93)/3 = 87$. Because $\bar x < 90$
      the experimenters will conclude that there is evidence that the diet
      helps people loose weight.
    \end{shaded}

    
  \item According to the central limit theorem, what is the asymptotic
    distribution of the sample mean of $n=3$ measurements from a population
    with mean $\mu=100$ and standard deviation $\sigma=15$?
    \begin{shaded}
      The central limit theorem says that the distribution of the sample mean
      $\bar X$ approximates the normal distribution with mean $\mu$ and
      variance $\sigma^2/n = 15^2/3 = 75$. In this case it will be the N($100, 75$)
      distribution. 
    \end{shaded}
    
  \item \label{item:1} Use the central limit theorem to calculate the probability of Type I error of the experimenters'
    decision rule.
    \begin{shaded}
      The experimenters' rule is ``reject H$_0$ if $\bar X < 90$''.

      $\Pr(\text{Type I error}) = \Pr(\bar X < 90|\mu = 100)$.

      Using the central limit theorem: $\bar X \sim \mathrm{N}(100, 75)$.
      Then to find the probability:
      \begin{center}
        \begin{minipage}{.45\linewidth}
          \includegraphics[width=\linewidth]{images/type1x.png}          
        \end{minipage}
        $\rightarrow$
        \begin{minipage}{.45\linewidth}
          \includegraphics[width=\linewidth]{images/type1z.png}  
          
        \end{minipage}
      \end{center}

      $z  = \dfrac{90-100}{\sqrt{75}} = -1.15 \Rightarrow \Phi(-1.15) =
      \mathbf{0.1251}$ is the probability of Type~I error.
    \end{shaded}

    
  \item Use the central limit theorem to calculate the probability of Type II error of the experimenters'
    decision rule assuming that the average weight after one month is 85.

    \begin{shaded}
      The experimenters' rule is ``reject H$_0$ if $\bar X < 90$''.

      $\Pr(\text{Type II error}) = \Pr(\bar X \geq 90|\mu = 85)$.

      Using the central limit theorem: $\bar X \sim \mathrm{N}(85, 75)$
      because now $\mu = 85$.
      Then to find the probability:
      \begin{center}
        \begin{minipage}{.45\linewidth}
          \includegraphics[width=\linewidth]{images/type2x.png}
        \end{minipage}
         $\rightarrow$        
        \begin{minipage}{.45\linewidth}
          \includegraphics[width=\linewidth]{images/type2z.png}  
        \end{minipage}
      \end{center}

      $z  = \dfrac{90-85}{\sqrt{75}} = 0.58 \Rightarrow \Phi(0.58) =
      0.7190 \Rightarrow 1 - 0.7190 = \mathbf{0.2810}$ is the probability of Type~II error.
    \end{shaded}

  \item Propose a rule of the form ``accept the author's claim if $\bar X <
    c$'' (in other words find $c$) such that the probability of Type~I
    error is 10\%.
    \begin{shaded}
      In this case we have to do the reverse calculation from~\ref{item:1}.
      If we want $\Phi(z) = 0.10$, then we must choose $z = -1.28$. In this
      case $\dfrac{c - 100}{\sqrt{75}} = -1.28 \Rightarrow c = 100 -(1.28)\sqrt{75} = \mathbf{88.9}$.
      \begin{center}
        \begin{minipage}{.45\linewidth}
          \includegraphics[width=\linewidth]{images/signz.png}
        \end{minipage}
         $\rightarrow$        
        \begin{minipage}{.45\linewidth}
          \includegraphics[width=\linewidth]{images/signx.png}  
        \end{minipage}
      \end{center}
    \end{shaded}

    \item Suppose that in the sample we find that $\bar x = 86$. Find the
    $p$-value. What is your conclusion at significance level $\alpha = 5\%$?
    \begin{shaded}
      \begin{center}
        \begin{minipage}{.45\linewidth}
          \includegraphics[width=\linewidth]{images/pval1x.png}
        \end{minipage}
         $\rightarrow$
        \begin{minipage}{.45\linewidth}
          \includegraphics[width=\linewidth]{images/pval1z.png}  
        \end{minipage}
      \end{center}
      $z  = \dfrac{86-100}{\sqrt{75}} = -1.62 \Rightarrow \Phi(-1.62) =
      \mathbf{0.0526}$ is the $p$-value.

      Since $\text{$p$-value} \nless \alpha$, we do not reject the null
      hypothesis at the 5\% level.
    \end{shaded}
  \end{enumerate}
\end{enumerate}
\end{document}



%%% Local Variables:
%%% mode: latex/m
%%% TeX-PDF-mode: t
%%% TeX-source-correlate-mode: t
%%% End:
