\documentclass[oneside,11pt]{article}

%\usepackage[british]{babel}
\usepackage{amsmath}
\usepackage{amssymb}
\usepackage{amsthm}
\usepackage{amsfonts}
\usepackage{graphicx}
% \usepackage{bm}
%\usepackage{dsfont}
%\usepackage{titling}
%\usepackage{enumitem}
%\usepackage{titletoc}
\usepackage{lmodern}
\usepackage{textcomp} % Required by listings
%\usepackage{listings}
%\usepackage{thmtools}
%\usepackage{memhfixc}
\usepackage{hyperref}
%\usepackage[all]{xy}


\usepackage[T1]{fontenc}
%\usepackage[scaled=0.8]{DejaVuSansMono}
% \usepackage[charter]{mathdesign}
% \def\rmdefault{bch} % not scaled
% \def\ttdefault{blg}
% \usepackage[scaled=0.9]{inconsolata}
% \usepackage[scaled=0.8]{luximono}

%\nonzeroparskip
\raggedbottom

%% Shaded environment
\usepackage{framed}
\usepackage{xcolor}
\definecolor{shadecolor}{gray}{0.9}
\setlength{\FrameSep}{3pt}
\setlength{\OuterFrameSep}{3pt}


%%% Floats
%\setfloatlocations{figure}{tbhp}
%\setfloatlocations{table}{tbhp}
%\captionnamefont{\bfseries}
% \captiontitlefont{\itshape}
% \captionstyle[\centering]{\raggedright}
% \captionwidth{\linewidth} \changecaptionwidth
\renewcommand{\topfraction}{0.9}        % max fraction of floats at top
\renewcommand{\bottomfraction}{0.8}     % max fraction of floats at bottom
\setcounter{topnumber}{2}
\setcounter{bottomnumber}{2}
\setcounter{totalnumber}{4}     % 2 may work better
\setcounter{dbltopnumber}{2}    % for 2-column pages
\renewcommand{\dbltopfraction}{0.9}     % fit big float above 2-col. text
\renewcommand{\textfraction}{0.07}      % allow minimal text w. figs
\renewcommand{\floatpagefraction}{0.7}  % require fuller float pages
\renewcommand{\dblfloatpagefraction}{0.7}       % require fuller float pages


%%% Commands
\DeclareMathOperator*{\E}{{\mathsf E}} % expectation
\DeclareMathOperator*{\Var}{{\mathsf Var}} % variance
\DeclareMathOperator*{\Cov}{{\mathsf Cov}} % covariance
\DeclareMathOperator*{\argmin}{argmin} % argmin
\DeclareMathOperator*{\argmax}{argmax} % argmax
\newcommand{\iid}{{{}\mathbin{\stackrel{\mathsf{iid}}{\sim}}{}}}
\renewcommand{\Pr}{{\mathbb{P}}}
\newcommand{\half}{{\frac 12}}
\newcommand{\ud}{{\,\mathrm{d}}}
\newcommand{\xbf}{{\mathbf{x}}}
\newcommand{\Xbf}{{\mathbf{X}}}

%%% Theorems
%\renewenvironment{proof}[1][Proof]{\noindent{\bfseries #1.}\hspace*{1em}}{\qed\\}
%\declaretheoremstyle[spaceabove=10pt,spacebelow=3pt,headfont=\normalfont\bfseries,
%bodyfont=\normalfont,postheadspace=1em,headpunct={.\newline}]{clearprint}
%\declaretheoremstyle[spaceabove=10pt,spacebelow=3pt,headfont=\normalfont\bfseries,
%bodyfont=\normalfont,postheadspace=1em,headpunct={.}]{clearexc}
%\declaretheoremstyle[spaceabove=10pt,spacebelow=3pt,headfont=\normalfont\bfseries,
%bodyfont=\normalfont,postheadspace=1em,headpunct={.},qed=$\blacktriangleright$]{clearexm}
%\declaretheoremstyle[spaceabove=10pt,headpunct={.\newline},
%headfont={\normalfont\itshape}]{remarkmy}
%%
%\declaretheorem[style=clearprint,within=chapter,name=Definition]{definition}
%\declaretheorem[style=clearexm,within=chapter,name=Example]{example}
%\declaretheorem[style=clearexc,within=chapter,name=Exercise]{exercise}
%\declaretheorem[style=clearprint,within=chapter,name=Theorem]{theorem}
%\declaretheorem[style=clearprint,name=Lemma,sibling=theorem]{lemma}
%\declaretheorem[style=clearprint,name=Proposition,sibling=theorem]{proposition}
%\declaretheorem[style=clearexc,numbered=no,name=Corollary]{corollary}
%\declaretheorem[style=clearexc,numbered=no,name=Remark]{remark}

\renewcommand{\labelitemi}{$\bullet$}
\setlength{\emergencystretch}{.5ex}

% \hypersetup{
%   colorlinks,%
%   citecolor=black,%
%   filecolor=black,%
%   linkcolor=black,%
%   urlcolor=black,
%   linktoc=none
% }

% Page style
\pagestyle{plain}

%%% Page size
% \setstocksize{297mm}{210mm} % A4 paper
% \settypeblocksize{258mm}{160mm}{*} % A4 paper
%\setstocksize{10.3in}{7.3in}
% \settypeblocksize{217mm}{140mm}{*}
%\settypeblocksize{230mm}{160mm}{*}
%\settrimmedsize{\stockheight}{\stockwidth}{*}
%\setlength{\trimtop}{0pt}
%\setlength{\trimedge}{\stockwidth}
%\addtolength{\trimedge}{-\paperwidth}
%\setulmargins{*}{*}{1.5}
%\setlrmargins{*}{*}{1}
\addtolength{\footskip}{.25cm}
%\checkandfixthelayout





\setcounter{tocdepth}{2}

\allowdisplaybreaks

\usepackage{xr}
\externaldocument[lec1:]{../1_sampling/lec1}
\externaldocument[lec2:]{../2_estimation/lec2}
\externaldocument[lec3:]{../3_hypothesis/lec3}



\begin{document}

Chapter 2 Solutions

\begin{enumerate}
\item A coffee shop buys roasted coffee from a supplier. In order to assess
  the quality of the supplied coffee, the manager of the shop conducts a
  tasting experiment where she selects a small portion of coffee beans from
  different batches and tastes the coffee from each portion. For each
  portion she gives a score in the scale $1,2,\ldots,10$ with 10
  corresponding to coffee of the best taste and uses the results to assess
  the quality of the coffee.

  Identify the population, parameter, and statistic.

  \begin{shaded}
    \textbf{Population:} In this case we want to draw conclusions about all
    roasted coffee delivered from the supplier.

    \textbf{Parameter:} In this tasting experiment we give a score to the
    taste from each batch. The parameter could be the average score from
    all possible batches even those which we didn't taste. Because each
    tasting experiment is a score from 1 to 10, the parameter space is the
    interval $[1,10]$.

    \textbf{Statistic:} The statistic that helps us estimate the value of
    the parameter is the sample average. The possible values of the
    statistic are in the range of 1 to 10. It is not easy to come up with
    the distribution of the test statistic. If we assume that the central
    limit theorem applies, then the distribution of the sample average is
    approximately the normal distribution.
  \end{shaded}

\item Read the abstract of the article: ``Dietary Intake of Marine n-3 Fatty Acids, Fish
  Intake, and the Risk of Coronary Disease among Men'' by Ascherio and
  others published in \textit{The New England Journal of Medicine} on April
  13, 1995\\ \url{http://www.nejm.org/doi/full/10.1056/NEJM199504133321501}

  Identify the population, parameter, sample, and statistic.

  \begin{shaded}
    \textbf{Population:} All men (according to the title).

    \textbf{Parameter:} There are two parameters of interest: (1)~The
    probability of having a coronary event (death from coronary disease,
    nonfatal myocardial infarction, and coronary-artery bypass or
    angioplasty procedures). (2)~The reduction, if any, of the risk of
    coronary disease caused by the intake of marine n-3 fatty acids.

    \textbf{Sample:} The sample consists of the 44,895 health
    professionals, who completed the questionnaire.
    
    \textbf{Statistic:} The number of participants in the study who
    experienced a coronary event.
  \end{shaded}

\item Let $X_1,\ldots,X_n \iid \mathrm{N}(\mu,\sigma^2)$. Derive the
  sampling distribution of $\bar X$ given in Example 2.4

  \begin{shaded}
  
Using the result in \href{https://moodle.bath.ac.uk/pluginfile.php/2857344/mod_resource/content/3/exercise-2-sols.html}{Problem 7  in Exercise sheet 2}, the moment generating function of a Normal distribution with mean $\mu$ and variance $\sigma^
2$ is given by:
$$
M_X(t) = e^{\mu t + \frac{\sigma^2 t^
2}{2}}
$$

Therefore the MGF of the average of n independent Normal distributions is:
$$
M_{\bar{X}}(t) = \left(M_X\left(\frac{
t}{n}\right)\right)^n = \left(e^{\mu \frac{t}{n} + \frac{\sigma^2 \left(\frac{t}{n}\right)^2}{2}}\right)^n = e^{\mu t + \frac{\sigma^2 t^2}{2n}}
$$
This is the MGF of a Normal distribution with mean $\mu$ and variance $\frac{\sigma^2}{n}$.
Thus, the average of n independent Normal distributions with mean $\mu$ and variance $\sigma^2$ is also Normally distributed with mean $\mu$ and variance $\frac{\sigma^2}{n}$.
 To find the
    mean and variance, we write $\bar X = \dfrac{1}{n} \sum_{i=1}^n X_i$ so
    \begin{align*}
      \E \bar X &= \E \left( \dfrac{1}{n} \sum_{i=1}^n X_i \right) &    \Var \bar X &= \Var \left( \dfrac{1}{n} \sum_{i=1}^n X_i \right) \\
      &= \dfrac{1}{n} \E \left( \sum_{i=1}^n X_i \right) &              &= \dfrac{1}{n^2} \Var \left( \sum_{i=1}^n X_i \right) \\
      &= \dfrac{1}{n} \sum_{i=1}^n \E X_i &                             &= \dfrac{1}{n^2} \sum_{i=1}^n \Var X_i,\ \text{by independence} \\
      &= \dfrac{1}{n} \sum_{i=1}^n \mu &                                &= \dfrac{1}{n^2} \sum_{i=1}^n \sigma^2 \\
      &= \dfrac{1}{n} n \mu  = \mu &                                    &= \dfrac{1}{n^2} n \sigma^2  = \sigma^2/n.
    \end{align*}
  \end{shaded}

\item Let $X_1,\ldots,X_n \iid \mathrm{Bernoulli}(p)$.
  \begin{enumerate}
  \item Derive the sampling distribution of $\bar X$.
  \item Derive the asymptotic distribution of $\bar X$ from the central
    limit theorem.
  \item Draw a graph of the exact and approximate CDFs when $n=20$ and $p =
    0.4$.
  \end{enumerate}

  \begin{shaded}
    \begin{enumerate}
    \item Let $W = \sum X_i$. Then $W \sim \mathrm{Bin}(n,p)$ and $\Pr(W =
      w) = \dbinom{n}{w} p^w (1-p)^{n-w}$.

      Therefore, $\Pr(\bar X = x) = \Pr(W = n x) = \dbinom{n}{nx} p^{nx}
      (1-p)^{n-nx}$. Thus, $\bar X$ is the sample proportion from $n$
      Bernoulli trials.

    \item The mean and variance of each $X_i$ is $\E X_i = p$ and $\Var X_i
      = p(1-p)$. By the central limit theorem $\bar X \sim N \left( p,
        \dfrac{p(1-p)}{n} \right)$ approximately.

    \item The two CDFs can be seen in the figure below\\
      \mbox{} \hfill \includegraphics[width=.65\linewidth]{images/exercise_cdf.png} \hfill
      \mbox{}\\
      with the red line denoting the approximate normal CDF and the blue
      line the exact binomial proportion. The exact CDF is a step function
      because the random variable is discrete.
      
Below is   some Python code to produce the plot.

\newpage
      
  \begin{verbatim}
import numpy as np
import matplotlib.pyplot as plt
from scipy.stats import binom, norm
n = 20
p = 0.4
x = np.arange(0, n+1)
binom_cdf = binom.cdf(x, n, p)
norm_cdf = norm.cdf(x, loc=n*p, scale=np.sqrt(n*p*(1-p)))
plt.step(x, binom_cdf, label='Binomial CDF', where='post
')
plt.plot(x, norm_cdf, label='Normal Approximation CDF', color='red
')
plt.xlabel('x')
plt.ylabel('CDF')
plt.title('Exact and Approximate CDFs')
plt.legend()
plt.grid()
plt.show()
\end{verbatim}
    \end{enumerate}
  \end{shaded}
\end{enumerate}
\end{document}



%%% Local Variables:
%%% mode: latex/m
%%% TeX-PDF-mode: t
%%% TeX-source-correlate-mode: t
%%% End:
