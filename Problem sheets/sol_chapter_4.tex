\documentclass[oneside,11pt]{article}

%\usepackage[british]{babel}
\usepackage{amsmath}
\usepackage{amssymb}
\usepackage{amsthm}
\usepackage{amsfonts}
\usepackage{graphicx}
% \usepackage{bm}
%\usepackage{dsfont}
%\usepackage{titling}
%\usepackage{enumitem}
%\usepackage{titletoc}
\usepackage{lmodern}
\usepackage{textcomp} % Required by listings
%\usepackage{listings}
%\usepackage{thmtools}
%\usepackage{memhfixc}
\usepackage{hyperref}
%\usepackage[all]{xy}


\usepackage[T1]{fontenc}
%\usepackage[scaled=0.8]{DejaVuSansMono}
% \usepackage[charter]{mathdesign}
% \def\rmdefault{bch} % not scaled
% \def\ttdefault{blg}
% \usepackage[scaled=0.9]{inconsolata}
% \usepackage[scaled=0.8]{luximono}

%\nonzeroparskip
\raggedbottom

%% Shaded environment
\usepackage{framed}
\usepackage{xcolor}
\definecolor{shadecolor}{gray}{0.9}
\setlength{\FrameSep}{3pt}
\setlength{\OuterFrameSep}{3pt}


%%% Floats
%\setfloatlocations{figure}{tbhp}
%\setfloatlocations{table}{tbhp}
%\captionnamefont{\bfseries}
% \captiontitlefont{\itshape}
% \captionstyle[\centering]{\raggedright}
% \captionwidth{\linewidth} \changecaptionwidth
\renewcommand{\topfraction}{0.9}        % max fraction of floats at top
\renewcommand{\bottomfraction}{0.8}     % max fraction of floats at bottom
\setcounter{topnumber}{2}
\setcounter{bottomnumber}{2}
\setcounter{totalnumber}{4}     % 2 may work better
\setcounter{dbltopnumber}{2}    % for 2-column pages
\renewcommand{\dbltopfraction}{0.9}     % fit big float above 2-col. text
\renewcommand{\textfraction}{0.07}      % allow minimal text w. figs
\renewcommand{\floatpagefraction}{0.7}  % require fuller float pages
\renewcommand{\dblfloatpagefraction}{0.7}       % require fuller float pages


%%% Commands
\DeclareMathOperator*{\E}{{\mathsf E}} % expectation
\DeclareMathOperator*{\Var}{{\mathsf Var}} % variance
\DeclareMathOperator*{\Cov}{{\mathsf Cov}} % covariance
\DeclareMathOperator*{\argmin}{argmin} % argmin
\DeclareMathOperator*{\argmax}{argmax} % argmax
\newcommand{\iid}{{{}\mathbin{\stackrel{\mathsf{iid}}{\sim}}{}}}
\renewcommand{\Pr}{{\mathbb{P}}}
\newcommand{\half}{{\frac 12}}
\newcommand{\ud}{{\,\mathrm{d}}}
\newcommand{\xbf}{{\mathbf{x}}}
\newcommand{\Xbf}{{\mathbf{X}}}

%%% Theorems
%\renewenvironment{proof}[1][Proof]{\noindent{\bfseries #1.}\hspace*{1em}}{\qed\\}
%\declaretheoremstyle[spaceabove=10pt,spacebelow=3pt,headfont=\normalfont\bfseries,
%bodyfont=\normalfont,postheadspace=1em,headpunct={.\newline}]{clearprint}
%\declaretheoremstyle[spaceabove=10pt,spacebelow=3pt,headfont=\normalfont\bfseries,
%bodyfont=\normalfont,postheadspace=1em,headpunct={.}]{clearexc}
%\declaretheoremstyle[spaceabove=10pt,spacebelow=3pt,headfont=\normalfont\bfseries,
%bodyfont=\normalfont,postheadspace=1em,headpunct={.},qed=$\blacktriangleright$]{clearexm}
%\declaretheoremstyle[spaceabove=10pt,headpunct={.\newline},
%headfont={\normalfont\itshape}]{remarkmy}
%%
%\declaretheorem[style=clearprint,within=chapter,name=Definition]{definition}
%\declaretheorem[style=clearexm,within=chapter,name=Example]{example}
%\declaretheorem[style=clearexc,within=chapter,name=Exercise]{exercise}
%\declaretheorem[style=clearprint,within=chapter,name=Theorem]{theorem}
%\declaretheorem[style=clearprint,name=Lemma,sibling=theorem]{lemma}
%\declaretheorem[style=clearprint,name=Proposition,sibling=theorem]{proposition}
%\declaretheorem[style=clearexc,numbered=no,name=Corollary]{corollary}
%\declaretheorem[style=clearexc,numbered=no,name=Remark]{remark}

\renewcommand{\labelitemi}{$\bullet$}
\setlength{\emergencystretch}{.5ex}

% \hypersetup{
%   colorlinks,%
%   citecolor=black,%
%   filecolor=black,%
%   linkcolor=black,%
%   urlcolor=black,
%   linktoc=none
% }

% Page style
\pagestyle{plain}

%%% Page size
% \setstocksize{297mm}{210mm} % A4 paper
% \settypeblocksize{258mm}{160mm}{*} % A4 paper
%\setstocksize{10.3in}{7.3in}
% \settypeblocksize{217mm}{140mm}{*}
%\settypeblocksize{230mm}{160mm}{*}
%\settrimmedsize{\stockheight}{\stockwidth}{*}
%\setlength{\trimtop}{0pt}
%\setlength{\trimedge}{\stockwidth}
%\addtolength{\trimedge}{-\paperwidth}
%\setulmargins{*}{*}{1.5}
%\setlrmargins{*}{*}{1}
\addtolength{\footskip}{.25cm}
%\checkandfixthelayout





\setcounter{tocdepth}{2}

\allowdisplaybreaks

\usepackage{xr}
\externaldocument[lec1:]{../1_sampling/lec1}
\externaldocument[lec2:]{../2_estimation/lec2}
\externaldocument[lec3:]{../3_hypothesis/lec3}




\begin{document}

Chapter 4 Solutions

\begin{enumerate}
\item Consider the method of moments estimator of Example 4.6
  \begin{enumerate}
  \item The following data were observed: $x_1 = 1$, $x_2 = 1$, $x_3 = 7$
    from the $\mathrm{U}(0,\theta)$ distribution.
    Compute the method of moments estimator for $\theta$.

  \begin{shaded}
    Given the data $x_1 = 1$, $x_2 = 1$, $x_3 = 7$ we find $\bar x = 3$ and so
    $\hat{\theta} = 2 \bar x = 6$.
  \end{shaded}
    
  \item Identify potential drawbacks of this estimator.

  \begin{shaded}
    This estimator can sometimes be nonsense. For example, using the data
    from the previous part, $\hat{\theta} = 6$. However we know that
    $\theta$ must be bigger than all $x_i$'s so $\theta > 7$ which
    contradicts $\hat{\theta} < 7$.
  \end{shaded}
\end{enumerate}

\item Verify the formula from the mgf of the gamma distribution in
  Example 4.7 and use it to derive its mean and variance.

  \begin{shaded}
    For the $\mathrm{Gamma}(\alpha,\beta)$ distribution,
    \begin{equation*}
      f(x|\alpha,\beta) = \dfrac{\beta^\alpha}{\Gamma(\alpha)} x^{\alpha-1}
      e^{-\beta x}.
    \end{equation*}
    Then
    \begin{align*}
      M_X(t)
      &= \int_0^\infty e^{tx} \dfrac{\beta^\alpha}{\Gamma(\alpha)} x^{\alpha-1}
        e^{-\beta x} \ud x \\
      &= \int_0^\infty \dfrac{\beta^\alpha}{\Gamma(\alpha)} x^{\alpha-1}
        e^{-(\beta-t) x} \ud x \\
      &= \dfrac{\beta^\alpha}{(\beta-t)^\alpha} \int_0^\infty \dfrac{(\beta-t)^\alpha}{\Gamma(\alpha)} x^{\alpha-1}
        e^{-(\beta-t) x} \ud x \\
      &= \dfrac{\beta^\alpha}{(\beta-t)^\alpha} ,\ \text{for $t < \beta$}.\\
      &= (1-t/\beta)^{-\alpha} ,\ \text{for $t < \beta$}.\\
      \Rightarrow M_X^{(1)}(t)
      &= \frac{d}{dt} M_X(t) = \dfrac{\alpha}{\beta} (1-t/\beta)^{-\alpha-1}\\
      \Rightarrow \mu_1
      &= M_X^{(1)}(0) = \dfrac{\alpha}{\beta} \\
      \Rightarrow M_X^{(2)}(t)
      &= \frac{d^2}{dt^2} M_X(t) = \dfrac{\alpha(\alpha+1)}{\beta^2} (1-t/\beta)^{-\alpha-2} \\
      \Rightarrow \mu_2
      &= M_X^{(2)}(0) = \dfrac{\alpha(\alpha+1)}{\beta^2} \\
      \Rightarrow \Var(X)
      &= \mu_2 - \mu_1^2 = \dfrac{\alpha(\alpha+1)}{\beta^2} -
        \dfrac{\alpha^2}{\beta^2} = \dfrac{\alpha}{\beta^2}.
    \end{align*}
  \end{shaded}


\item Explain why the MLE of Example 4.10 is biased but do not
  derive its bias.  \textit{Hint.} Think why $\max\{X_i\} < \theta$, and what
  this means about $\E [ \max\{X_i\} ]$.

  \begin{shaded}
    Because each $X_i < \theta$, then $X_{(n)} = \max\{X_1,\ldots,X_n\} <
    \theta$, so $\E X_{(n)} < \theta$. Therefore
    $\mathrm{Bias}_\theta(X_{(n)}) = \E X_{(n)} - \theta < 0$. So
    $\mathrm{Bias}_\theta(X_{(n)}) \neq 0$ which means that $X_{(n)}$ is
    biased for $\theta$.
  \end{shaded}

\item Let $X_1,\ldots,X_n \iid \mathrm{Bernoulli}(\theta)$,
  $\theta \in (0,1)$.
  \begin{enumerate}
  \item Derive the MLE for $\theta$.

  \begin{shaded}
    The pmf for $X$ is
    \begin{equation*}
      f(x|\theta) = \theta^x (1-\theta)^{1-x},\ x \in \{0,1\},\ \theta \in
      [0,1].
    \end{equation*}
    Taking logarithms,
    \begin{align*}
      \log f(x|\theta)
      &= x \log \theta + (1-x) \log (1-\theta) \\
      \Rightarrow \ell(\theta|\xbf)
      &= \sum_{i=1}^n \log f(x_i|\theta) \\
      &= \sum_{i=1}^n \left\{ x_i \log \theta + (1-x_i) \log (1-\theta)
        \right\} \\
      &= \log \theta \sum_{i=1}^n x_i  + \log (1-\theta) (n - \sum_{i=1}^n
        x_i) \\
      \Rightarrow \ell'(\theta|\xbf)
      &= \frac 1 \theta \sum_{i=1}^n x_i  - \frac{1}{1-\theta} (n - \sum_{i=1}^n
        x_i) \\
      0
      &= \frac{1}{ \hat \theta} \sum_{i=1}^n x_i  - \frac{1}{1-\hat\theta} (n
        - \sum_{i=1}^n  x_i) \\
      0
      &= \frac{n}{ \hat \theta} \bar x  - \frac{n}{1-\hat\theta} (1
        - \bar x) \\
      0
      &= \frac{\bar x}{ \hat \theta}  - \frac{1-\bar x}{1-\hat\theta} \\
      \hat{\theta} &= \bar x
    \end{align*}
    Evaluating the second derivative of $ \ell(\theta|\xbf)$ at $\theta=\bar{x}$ is negative which confirms this is a maximum.
    
  \end{shaded}
\end{enumerate}


\end{enumerate}
\end{document}



%%% Local Variables:
%%% mode: latex/m
%%% TeX-PDF-mode: t
%%% TeX-source-correlate-mode: t
%%% End:
