\documentclass[oneside,11pt]{article}

%\usepackage[british]{babel}
\usepackage{amsmath}
\usepackage{amssymb}
\usepackage{amsthm}
\usepackage{amsfonts}
\usepackage{graphicx}
% \usepackage{bm}
%\usepackage{dsfont}
%\usepackage{titling}
%\usepackage{enumitem}
%\usepackage{titletoc}
\usepackage{lmodern}
\usepackage{textcomp} % Required by listings
%\usepackage{listings}
%\usepackage{thmtools}
%\usepackage{memhfixc}
\usepackage{hyperref}
%\usepackage[all]{xy}


\usepackage[T1]{fontenc}
%\usepackage[scaled=0.8]{DejaVuSansMono}
% \usepackage[charter]{mathdesign}
% \def\rmdefault{bch} % not scaled
% \def\ttdefault{blg}
% \usepackage[scaled=0.9]{inconsolata}
% \usepackage[scaled=0.8]{luximono}

%\nonzeroparskip
\raggedbottom

%% Shaded environment
\usepackage{framed}
\usepackage{xcolor}
\definecolor{shadecolor}{gray}{0.9}
\setlength{\FrameSep}{3pt}
\setlength{\OuterFrameSep}{3pt}


%%% Floats
%\setfloatlocations{figure}{tbhp}
%\setfloatlocations{table}{tbhp}
%\captionnamefont{\bfseries}
% \captiontitlefont{\itshape}
% \captionstyle[\centering]{\raggedright}
% \captionwidth{\linewidth} \changecaptionwidth
\renewcommand{\topfraction}{0.9}        % max fraction of floats at top
\renewcommand{\bottomfraction}{0.8}     % max fraction of floats at bottom
\setcounter{topnumber}{2}
\setcounter{bottomnumber}{2}
\setcounter{totalnumber}{4}     % 2 may work better
\setcounter{dbltopnumber}{2}    % for 2-column pages
\renewcommand{\dbltopfraction}{0.9}     % fit big float above 2-col. text
\renewcommand{\textfraction}{0.07}      % allow minimal text w. figs
\renewcommand{\floatpagefraction}{0.7}  % require fuller float pages
\renewcommand{\dblfloatpagefraction}{0.7}       % require fuller float pages


%%% Commands
\DeclareMathOperator*{\E}{{\mathsf E}} % expectation
\DeclareMathOperator*{\Var}{{\mathsf Var}} % variance
\DeclareMathOperator*{\Cov}{{\mathsf Cov}} % covariance
\DeclareMathOperator*{\argmin}{argmin} % argmin
\DeclareMathOperator*{\argmax}{argmax} % argmax
\newcommand{\iid}{{{}\mathbin{\stackrel{\mathsf{iid}}{\sim}}{}}}
\renewcommand{\Pr}{{\mathbb{P}}}
\newcommand{\half}{{\frac 12}}
\newcommand{\ud}{{\,\mathrm{d}}}
\newcommand{\xbf}{{\mathbf{x}}}
\newcommand{\Xbf}{{\mathbf{X}}}

%%% Theorems
%\renewenvironment{proof}[1][Proof]{\noindent{\bfseries #1.}\hspace*{1em}}{\qed\\}
%\declaretheoremstyle[spaceabove=10pt,spacebelow=3pt,headfont=\normalfont\bfseries,
%bodyfont=\normalfont,postheadspace=1em,headpunct={.\newline}]{clearprint}
%\declaretheoremstyle[spaceabove=10pt,spacebelow=3pt,headfont=\normalfont\bfseries,
%bodyfont=\normalfont,postheadspace=1em,headpunct={.}]{clearexc}
%\declaretheoremstyle[spaceabove=10pt,spacebelow=3pt,headfont=\normalfont\bfseries,
%bodyfont=\normalfont,postheadspace=1em,headpunct={.},qed=$\blacktriangleright$]{clearexm}
%\declaretheoremstyle[spaceabove=10pt,headpunct={.\newline},
%headfont={\normalfont\itshape}]{remarkmy}
%%
%\declaretheorem[style=clearprint,within=chapter,name=Definition]{definition}
%\declaretheorem[style=clearexm,within=chapter,name=Example]{example}
%\declaretheorem[style=clearexc,within=chapter,name=Exercise]{exercise}
%\declaretheorem[style=clearprint,within=chapter,name=Theorem]{theorem}
%\declaretheorem[style=clearprint,name=Lemma,sibling=theorem]{lemma}
%\declaretheorem[style=clearprint,name=Proposition,sibling=theorem]{proposition}
%\declaretheorem[style=clearexc,numbered=no,name=Corollary]{corollary}
%\declaretheorem[style=clearexc,numbered=no,name=Remark]{remark}

\renewcommand{\labelitemi}{$\bullet$}
\setlength{\emergencystretch}{.5ex}

% \hypersetup{
%   colorlinks,%
%   citecolor=black,%
%   filecolor=black,%
%   linkcolor=black,%
%   urlcolor=black,
%   linktoc=none
% }

% Page style
\pagestyle{plain}

%%% Page size
% \setstocksize{297mm}{210mm} % A4 paper
% \settypeblocksize{258mm}{160mm}{*} % A4 paper
%\setstocksize{10.3in}{7.3in}
% \settypeblocksize{217mm}{140mm}{*}
%\settypeblocksize{230mm}{160mm}{*}
%\settrimmedsize{\stockheight}{\stockwidth}{*}
%\setlength{\trimtop}{0pt}
%\setlength{\trimedge}{\stockwidth}
%\addtolength{\trimedge}{-\paperwidth}
%\setulmargins{*}{*}{1.5}
%\setlrmargins{*}{*}{1}
\addtolength{\footskip}{.25cm}
%\checkandfixthelayout





\setcounter{tocdepth}{2}

\allowdisplaybreaks

\usepackage{xr}
\externaldocument[lec1:]{../1_sampling/lec1}
\externaldocument[lec2:]{../2_estimation/lec2}
\externaldocument[lec3:]{../3_hypothesis/lec3}



\begin{document}

Chapter 3 Solutions

\begin{enumerate}
\item A patient is considering a number of treatment options available
  through her general practitioner (GP) between receiving medication or
  having a surgery. The costs of the different treatments vary as well
  as their likelihood of success. The GP has discussed with the
  patient the success rates of each treatment when used in other patients.

  Describe the parameter, data, actions, and loss function for this
  problem.

  \begin{shaded}
    The \textbf{parameter} corresponds to the probability of success of
    each treatment.

    The \textbf{data} consist of the success rates  provided by the GP.

    The available \textbf{actions} are: no treatment, medication, surgery.

    The \textbf{loss} function incorporates the cost and improvement in
    quality of life for the patient.
  \end{shaded}
  
\item An investor is considering whether or not to buy certain risky bonds.
  If he buys the bonds, they can be redeemed at maturity for a net gain of
  \pounds500. There is probability $\theta$ that there will be a default
  on the bonds, in which case the investor is set to lose his investment of
  \pounds1000. If the investor instead puts his money in a ``safe''
  investment, he will receive a net gain of \pounds300 over the same
  period.
  \begin{enumerate}
  \item Define appropriate actions, parameter, and parameter space for the problem.
    \begin{shaded}
      The two \textbf{actions} are: ``Invest in bonds'' and ``Invest in
      safe assets''. The \textbf{parameter} is the probability that there
      is a default on the bonds. The \textbf{parameter space} is $[0,1]$.
    \end{shaded}
  \item Derive the loss function for the problem.
    \begin{shaded}
      Let $y$ denote the event of default on the bonds. We set $y=1$ if a
      default occurs and $y=0$ if a default does not occur. Then $y \sim
      \mathrm{Bernoulli}(\theta)$. Let $a=1$ if the investor invests in
      bonds and $a=0$ if the investor invests in safe assets. Then, the
      loss from the outcome $y$ is\\
      \begin{equation*}
        l(y,a) =
        \begin{cases}
          -300 & \text{ if $a=0$ and $y=0$ or $1$}, \\
          -500 & \text{ if $a=1$ and $y=0$,} \\
          1000 & \text{ if $a=1$ and $y=1$.}
        \end{cases}
      \end{equation*}
      This can also be written as a $2\times 2$ table:
      \begin{center}
        \begin{tabular}{l|c|c|}
          \multicolumn{1}{c}{} & \multicolumn{1}{c}{$a=0$} &
          \multicolumn{1}{c}{$a=1$} \\
          \cline{2-3}
          $y=0$ & $-300$ & $-500$ \\
          \cline{2-3}
          $y=1$ & $-300$ & $1000$ \\
          \cline{2-3}
        \end{tabular}.
      \end{center}
      Then,
      \begin{align}
        L(\theta,a=0) &= \E[l(y,a=0)] \\ &= l(y=0,a=0) \Pr(y=0) +
        l(y=1,a=0) \Pr(y=1) \\ &= -300 \times (1-\theta) -300 \times \theta
        \\ &= -300 \\
        L(\theta,a=1) &= \E[l(y,a=1)] \\ &= l(y=0,a=1) \Pr(y=0) +
        l(y=1,a=1) \Pr(y=1) \\ &= -500 \times (1-\theta) +1000 \times \theta
        \\ &= -500 + 1500 \theta
      \end{align}
      So,
      \begin{equation*}
        L(\theta,a) =
        \begin{cases}
          -300 & \text{ if $a=0$,}\\ -500 + 1500\theta & \text{ if $a=1$.}
        \end{cases}
      \end{equation*}
    \end{shaded}
  \item Describe all randomised decision rules and find the minimax
    decision among them.
    \begin{shaded}
      There are no data. All randomised decision rules are of the form
      \begin{equation*}
        d_p =
        \begin{cases}
          1 & \text{ with probability $p$,} \\ 0 & \text{ with probability
            $1-p$,} 
        \end{cases}
      \end{equation*}
      i.e., invest in bonds with probability $p$, and invest in safe assets
      with probability $1-p$. As there are no data, the risk is the loss, i.e.,
      \begin{align*}
        R(\theta,d_p) &= L(\theta,d_p) \\
        &= -300\times(1-p) + (-500 + 1500\times\theta) \times p \\
        &= -300 - 200\times p + 1500\times\theta\times p.
      \end{align*}
      This is maximised when $\theta=1$ for all $p$, so
      \begin{align*}
        \bar R(d_p) &= \max_{0 \leq \theta \leq 1} R(\theta,d_p) \\
        &= -300 -200 \times p + 1500 \times 1 \times p \\
        &= -300 + 1300 \times p.
      \end{align*}
      The value of $p$ that minimises the maximum risk, $\bar R(d_p)$ is
      $p=0$, so the optimal decision is in fact a deterministic decision, 
      \begin{equation*}
        d_\mathrm{MM} = d_0 =
        \begin{cases}
          1 & \text{ with probability $0$,} \\ 0 & \text{ with probability
            $1$} 
        \end{cases} = 0 \text{ always}.
      \end{equation*}
      In other words, the minimax strategy is to invest in safe assets.
    \end{shaded}
  \end{enumerate}


\item A coin has probability $\theta \in [0,1]$ of coming up heads $(y=1)$,
  and $1-\theta$ of coming up tails $(y=0)$. You are playing a game where
  if you guess the outcome of a coin flip correctly you receive a payment
  of \pounds{1}, but if you guess wrongly, you loose \pounds{1}.
  \begin{enumerate}
  \item What are the parameter and parameter space for this problem?
    \begin{shaded}
      The parameter is the probability of the coin coming up heads,
      $\theta$, and the parameter space is $\Theta = [0,1]$.
    \end{shaded}
    
  \item What is the action space for this problem?

    \begin{shaded}
      The actions that can be taken are: ``guess heads'' ($a=1$) or ``guess
      tails'' ($a=0$), so the action space is $\mathcal{A} = \{0,1\}$.
    \end{shaded}
    
  \item Show that the loss function, $L(\theta,a)$, for this problem is given
    by
    \begin{equation*}
      L(\theta,a) =
      \begin{cases}
        2\theta -1 & \text{ if guessing ``tails'',} \\ 1-2\theta  & \text{ if guessing ``heads''.}
      \end{cases}
    \end{equation*}

    \begin{shaded}
      Let $y$ be the outcome of a future coin flip such that $\Pr(y=1) =
      \theta$ and $\Pr(y=0) = 1-\theta$. Then the loss from that outcome is
      \begin{align*}
        l(y,0) &= \begin{cases} -1 & \text{ if $y=0$,} \\ 1 & \text{ if
            $y=1$,} 
        \end{cases}
        & l(y,1) &= \begin{cases} 1 & \text{ if $y=0$,} \\ -1 &
          \text{ if $y=1$.}
        \end{cases}
      \end{align*}
      Then, $L(\theta,0) = -1 \times \Pr(y=0) + 1 \times \Pr(y=1) =
      -(1-\theta) + \theta = 2\theta-1$,\\
      and $L(\theta,1) = 1 \times \Pr(y=0) - 1 \times \Pr(y=1) =
      (1-\theta) - \theta = 1-2\theta$.
    \end{shaded}
    
    \end{enumerate}
    
  \item Let $x$ be the outcome the coin flip from an earlier game. Consider
    the following two strategies for guessing the outcome of a future coin
    flip:
    \begin{itemize}
    \item \textbf{Strategy 1:} Guess the same as the outcome of the earlier
      coin flip. 
    \item \textbf{Strategy 2:} Guess ``heads'' regardless of the outcome of
      the earlier coin flip.
    \end{itemize}
    \begin{enumerate}
    \item Write a mathematical expression for the decision rules
      corresponding to these two strategies and compute their risks.

      \begin{shaded}
        The decision rules corresponding to the two strategies are $d_1(x)
        = x$ and $d_2(x) = 1$.\medskip

        Since $x = 1$ with probability $\theta$, and $x=0$ with probability
        $1-\theta$, \\ $R(\theta,d_1) = \E L(\theta,x) =
        L(\theta,0)(1-\theta) + L(\theta,1)\theta = (2\theta-1)(1-\theta) +
        (1-2\theta) \theta = (2\theta-1)(1-\theta) -
        (2\theta-1) \theta = (2\theta-1)(1-\theta-\theta) =
        (2\theta-1)(1-2\theta) = -(2\theta-1)^2$. \medskip

        $R(\theta,d_2) = \E L(\theta,1) = \E(1-2\theta) = 1-2\theta$.
      \end{shaded}
      
    \item Between the two strategies, which one is the minimax decision
      rule?

      \begin{shaded}
        It can be seen that for $d_1$, the value of $\theta$ that attains
        the maximum risk is $\theta = \half$, in which case $\bar R(d_1) =
        -(2 \times \half - 1)^2 = 0$.

        For $d_2$, the risk is maximised when $\theta = 0$, in which case
        $\bar R(d_2) = 1 - 2 \times 0 = 1$.

        So, between the two decision rules, $d_1$ is minimax as $\bar
        R(d_1) < \bar R(d_2)$.
      \end{shaded}
 
      

  \end{enumerate}

% \item Every winter evening, the head of Bath's council must decide whether
%   to spread salt on the roads of Bath to prevent them from getting icy. If
%   she does not spread salt, and the roads become icy, then there are
%   consequences from car accidents and delays in the roads. Spreading salt,
%   also costs money. She calculates that the cost from the roads becoming
%   icy if no salt is spread is twice as much as the cost of spreading salt.
%   There is no loss from rightly spreading salt.
% 
%   To make up her decision, the head of council checks two weather reporting
%   websites that provide independent forecasts of the event of ice. If there
%   is going to be ice, each website will forecast ice with probability
%   $3/4$, and not forecast ice with probability $1/4$. If there is not going
%   to be ice, each website will forecast ice with probability $1/2$, and not
%   forecast ice with probability $1/2$. Historical records show that a third
%   of winter nights are icy. Let $x$ denote how many of the two
%   websites forecast ice on a given day. The head of council will make a
%   decision based on $x$.
% 
%   \begin{enumerate}[label=\alph*)]
%   \item Let $\theta$ denote the probability that there will be ice on any
%     particular day. Show that the loss function for this problem is given by
%     \begin{equation*}
%       L(\theta,a) = \begin{cases} 2\theta & \text{ if salt is not spread,}
%         \\ 1-\theta & \text{ if salt is spread.} \end{cases}
%     \end{equation*}
% 
%     \begin{shaded}
%       There is a loss of 1 if salt is spread but the weather is not icy,
%       and a loss of 2 if salt is not spread but the weather is icy. In all
%       other cases, there is no loss. Let $a=1$ if salt is spread, and $a=0$
%       if salt is not spread. Then,
%       \begin{align*}
%         L(\theta,0) &= 0 \times \Pr(\text{no ice}) + 2 \times
%         \Pr(\text{ice}) = 0 \times (1-\theta) + 2 \times \theta = 2\theta,
%         \\
%         L(\theta,1) &= 1 \times \Pr(\text{no ice}) + 0 \times
%         \Pr(\text{ice}) = 1 \times (1-\theta) + 0 \times \theta = 1-\theta.
%       \end{align*}
%     \end{shaded}
%     
%   \item Write down an exhaustive set of deterministic decision rules based
%     on $x$. \label{item:3}
% 
%     \begin{shaded}
%       Here, $x$ can take the values $x\in\{0,1,2\}$ and each value can be
%       assigned to an action $a \in \{0,1\}$. Therefore, there are $2^3=8$
%       different deterministic decision rules.
% 
%       The different decision rules are to spread salt if $x$ is a member of
%       the set: $\{\}$, $\{0\}$, $\{1\}$, $\{2\}$, $\{0,1\}$, $\{0,2\}$,
%       $\{1,2\}$, $\{0,1,2\}$, and not spread otherwise. The set $\{\}$
%       correspond to never spread salt (regardless of $x$), and the set
%       $\{0,1,2\}$ to always spread salt (regardless of $x$).
%     \end{shaded}
%     
%   \item Compute the risk of each rule from~\ref{item:3} and find the minimax
%     decision rule among them.
%   \item Consider the following family of randomised decision rules,
%     $d_p(x)$, where salt is spread with probability $px$, where $0 \leq p
%     \leq \half$.
%     \begin{enumerate}
%     \item Find the minimax decision rule in $d_p(x)$.
%     \item Find the Bayes decision rule under the uniform prior $\pi(\theta)
%       = 1$.
%     \end{enumerate}
%   \end{enumerate}
\end{enumerate}
\end{document}



%%% Local Variables:
%%% mode: latex/m
%%% TeX-PDF-mode: t
%%% TeX-source-correlate-mode: t
%%% End:
